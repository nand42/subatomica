\section{Decadimenti nucleari}
La formula semi-emipirica di massa, equazione \ref{formula_semiempirica_massa} nei capitoli precedenti, esprime la stabilità o instabilità dei nuclei nel caso in cui ci sia un'abbondanza di neutroni o di protoni.
Per basi numeri atomici il numero di protoni e di neutroni si equivale, aumentando il numero di massa i nuclei tendono ad avere più neutroni che protoni ed in particolare i nuclei stabili e arrivano ad un numero atomico di circa 80 ed un numero di neutroni di circa 110.
Tutti gli atomi con elevato numero atomico e di massa sono instabili, come ad esempio l'uranio che ha numero atomico 92 ed è un elemento instabile che si trova in natura.










