%Zarantonello Umberto 20/04/21

\section{Il Nucleo}
La secoda particella ad essere rivelata dopo l'elettrone fu il protone. 
La prima evidenza sperimentale è dovuta a Rutherford nel 1917 sfruttando sempre le particelle alfa, questa volta in collisione con l'aria dove è presente l'azoto N 
\begin{equation}
^4_2He +^{14}_{7}N_{7}\longrightarrow ^{17}_8O_9+protone
\end{equation}
Al tempo uno dei modelli nucleari era che il nucleo fosse composto da un certo numero $A$ di protoni ed un numero $A-Z$ di elettroni (con Z il numero di elettroni della nuvola elettronica) in modo da bilanciare la carica dei protoni rendendo l'atomo stabile.
Analizziamo dunque la possibilità di avere un nucleo di questo tipo.
Che potenziale elettroagnetico dovrebbero gestire i protoni?
\[E=\frac{Ze^2}{4\pi\varepsilon_0R}\]
dove $R=R_0 A^{1/3}=1.2fm\cdot A^{1/3}$ (formula empirica per il calcolo del raggio nucleare). 
Calcolando si ottiene quindi
\[E=-1.20\frac{Z}{A^{1/3}MeV}\]
Per esempio, ponendo A=140 e Z=58 l'energia cinetica che si ottiene è $E=-13.4MeV$ (elettrone con energia relativistica quindi $E=pc$). Un elettrone con tale energia cinetica è possibile che resti confinato all'interno del nucleo?
\[\lambda=\frac{h}{p}=2\pi \frac{\hbar c}{pc}=\frac{2\cdot 3.14\cdot 200MeV\cdot fm}{13.4MeV}\simeq 90fm\]
L'elettrone non può essere quindi confinato nel nucleo perchè la sua lunghezza d'onda è molto maggiore.

\paragraph{La scoperta del Neutrone} fu fatta da Chadwick che fu il primo ad intuire, da un'esperienza che in realtà era già stata fatta da più fisici, la presenza di un'altra particella. 
La situazione che si verificava era che tramite l'emissione di particelle alfa generate dal polonio, fatte collidere su un target di Berilio, si otteneva una radiazione che riusciva ad attraversare uno schermo spesso di piombo, il che suggeriva il fatto che fosse una radiazione neutra (impossibile per della radiazione carica attraversare uno schermo troppo spesso). 
Al tempo l'unica radiazione neutra conosciuta era la radiazione elettromagnetica il che fece pensare ad una reazione cle tipo
\[
^9_4Be+^4_2He\longrightarrow[^{13}_6C*]\longrightarrow^{13}_6C+\gamma
\]
Con uno stato intermedio dato da uno stato eccitato del carbonio 13.
Un ulteriore passaggio fu quello di aggiungere dopo lo schermo di Piombo $Pb$ una lastra di paraffina da cui, dopo l'interazione con la radiazione, emergevano protoni con energia pari a $E=7,5MeV$. 
La radiazione gamma doveva quindi possedere un'energia in grado di podurre dei protoni di energia 7,5MeV trammite scattering Compton il che riconduce ad un'energia minima di 55MeV.
Quando il Berilio assorbiva le particelle alfa quest'ultime si trovavano ad un energia di 5MeV, essendo poi una reazione esotermica si aveva che il Q della reazione corrispondeva a $Q=10MeV$, il che riconduceva ad un energia massima disponibile di 14MeV.
L'unica spiegazione possibile era che si trattava quindi di una particella nuova, neutra e con la stessa massa del protone.
