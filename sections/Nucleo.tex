\section{Il Nucleo}
Il nucleo è composto da protoni e neutroni
Rivelazione del Protone
La scoperta del protone avviene da parte di Rutherford nel 1917 sfruttando sempre le particelle alfa
\[^4_2He +^{14}_{7}N_{7}\to ^{17}_8O_9+protone\]
Al tempo una delle ipotesi per il nucleo era che fosse fatto da un certo numero A di protoni ed un numero Z di elettroni in modo da bilanciare la carica dei protoni e fare in modo che il tutto fosse legato. Analizziamo dunque la possibilità di avere un nucleo di questo tipo.
Che potenziale elettroagnetico dovrebbero gestire i protoni?
\[E=\frac{Ze^2}{4\pi\varepsilon_0R}\]
dove $R=R_0 A^{1/3}=1.2fm\cdot A^{1/3}$ (formula empirica per il calcolo del raggio nucleare. Calcolando si ottiene quindi
\[E=-1.20\frac{Z}{A^{1/3}MeV}\]
Per esempio, ponendo A=140 e Z=58 l'energia cinetica che si ottiene è $E=-13.4MeV$ (elettrone con energia relativistica quindi $E=pc$). Un elettrone con tale energia cinetica è possibile che resti confinato all'interno del nucleo?
\[\lambda=\frac{h}{p}=2\pi \frac{\hbar c}{pc}=\frac{2\cdot 3.14\cdot 200MeV\cdot fm}{13.4MeV}\simeq 90fm\]
L'elettrone non può essere quindi confinato nel nucleo perchè la sua lunghezza d'onda è molto maggiore.