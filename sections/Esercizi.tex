%Zarantonello Umberto 22/04/21
\section{Esercizi}
\subsection{Settimana 1}
\paragraph{Es. 1:}
Si calcoli la velocità media di:
\begin{itemize}
\item Molecole d'aria a temperatura ambiente
\item Eletroni atomo idrogeno
\item Terra attorno al sole
\item Elettroni che escono da un vecchio tubo catodico
\end{itemize}
\paragraph{Ris.:}
\begin{itemize}
\item \textbf{Molceole d'aria}
supponiamo di essere a $20^oC$ corrispondenti a $293K$ si sfrutti la teoria cinetica dei gas
\begin{equation}
\frac{3}{2}K_BT=\frac{1}{2}m<v^2>
\end{equation}
Si approssimi l'aria come azoto $N_2$ la cui massa molecolare è $M_{N_2}=28$
si ottiene
\begin{equation}
\sqrt{<v^2>}=\sqrt{\frac{3K_BT}{m_{N_2}}}=\sqrt{\frac{3\times1,3\times10^{-23}293}{28\cdot 1,66\times10^{-27}kg}}=510\frac{m}{s}
\end{equation}
\item \textbf{Elettroni dell'idrogeno}
Ricordando la costante di Ridberg ovvero il potenziale di dissociazione dell'idrogeno, questa può essere considerata pari alla sua energia cinetica.
\begin{equation}
E=13.5eV=K_e\\
K=135\times10\cdot1,6\times10^{-19}J=2,1\times10^{-18}J=\frac{1}{2}m_ev^2
\end{equation}
conoscendo la massa dell'elettrone corrispondente a $m_e=9,1\times10^{-31}kg$ si ottiene che la velocità dell'elettrone sarà
\begin{equation}
v_e=\sqrt{\frac{2K_e}{m_e}}=2,1\times10^6\frac{m}{s}
\end{equation}
Si può notare che questa è una conferma che l'elettrone sia una particella non relativistica.
\end{itemize}
Si lasciano al lettore gli altri due punti.

\newpage
\paragraph{Es. 2:}
Si calcoli il numero di molecole nell'atmosfera.
\paragraph{Ris.:}
Si parte considerando la massa dell'atmosfera, che si può calcolare partendo dalla pressione atmosferica, corrispondente al dell'aria sopra un metro quadro.
\begin{equation}
p=10^5\frac{N}{m^2}\longrightarrow M=10^4\frac{kg}{m^2}
\end{equation}
Si consideri, ora la superficie della terra 
\begin{equation}
A_{terra}=4\pi R^2=12(6\times10^6m)^4=4\times10^14m^2
\end{equation}
Si può quindi trovare la massa dell'aria
\begin{equation}
M_{aria}=10^4\frac{kg}{m^2}4\times10^{14}m^2=4\times10^{18}kg
\end{equation}
Per calcolare il numero di molecoe di aria, si consideri il pero molecolare di azoto e ossigeno
\begin{equation}
N_2=28\hspace{0,5cm}O_2=32
\end{equation}
Il che in media corrisponde ad un peso molecolare di $30$.
Una mole peserà dunque $30g$
\begin{equation}
M_{aria}=\frac{4\times10^21g}{30g/mol}=1,3\times10^{20}mol
\end{equation}
Il numero di molecole d'aria corrispondera quindi alla massa in moli dell'aria moltiplicata per il numero di Avogadro $N_A=6\times10^{23}$
\begin{equation}
N_{aria}=M_{aria}\times N_A\simeq 10^41molecole
\end{equation}
Se si volesse poi sapere quante molecole di aria dell'ultimo respiro di Carlo Magno sono contenute nei nostri polmoni, si dovrebbe fare il rapporto fra la quantità di aria contenuta nei nostri polmoni e quella contenuta nell'atmosfera.
\begin{equation}
1mole (STP)=20L
\end{equation}
Una mole in condizioni standard corrisponde a 20 litri, la densità dell'aria corrisponderà quindi a 
\begin{equation}
30\frac{g}{mol}:20\frac{L}{mol}=1,5\frac{g}{L}\\
1L\sim 1g\sim 3\times 10^{22}molecole
\end{equation}
Qual è la frazione di molecole che noi respiriamo ad ogni respiro?
La massa dell'aria totale è pari a $4\times 10^21g$ e ogni respiro corrsponde ad un peso di circa $1g$, il che restituisce un rapporto di
\begin{equation}
0,25\times 10^{-21}
\end{equation}

\newpage
\paragraph{Es. 3:}
si calcoli l'energia contenuta in un $kg$ di benzina, pprossimando la benzina come $CH_2$.
\paragraph{Ris.:}
Si può stimare che per ogni legame chimico l'energia sia pari a $E=1,5eV$ (\'E una stima molto approssimata).
Si considerino le masse atomiche:

Il carbonio ha massa atomica $C=12$, mentre l'idrogeno $H=1$, il che riconduce ad una massa totale pari a $CH_2=14$.

In un $Kg$ di benzina si ha
\begin{equation}
N_{moli} =\frac{1Kg}{1,4\times 10^{-2}Kg/mol}=70mol
\end{equation}
Per ogni molecola di $CH_2$ avrò due reazioni
\begin{equation}
C+O_2\longrightarrow CO_2\\
H_2+O\longrightarrow H_2O
\end{equation}
Corrispondente ad un'energia totale rilasciata di $3eV$.

Qual'è la densità energetica della benzina?
\begin{equation}
D=70\frac{mol}{Kg}\cdot 6\times 10^{23}\frac{reazioni}{mol}\frac{3eV}{mol}\cdot \frac{1}{1,6\times 10^{-19}eV/J}=2\times 10^7\frac{J}{Kg}
\end{equation}
Questo valore è approssimativo ma si discosta solamente di un fattore 2 dal valore reale, il che ci f intuire che comunque si tratta di un buon calcolo (che il bravo fisico deve essere in grado di effettuare).

Qual è poi la potenza trasferita in un pieno?

Supponiamo che un serbatoio di un'auto di $80L$. La densità della benzina è più bassa di quella dell'acqua. 
La densità di energia per litro corrisponde a $3\times10^7J/L$
\begin{equation}
80L\cdot 3\times10^7\frac{J}{L}=2\times10^9J
\end{equation}
In 3 minuti (tempo di un pieno) l'energia trasferita corrisponde a 
\begin{equation}
P=\frac{E}{\Delta t}=\frac{2\times10^9J}{180s}=10MW
\end{equation}
Impressionante!

