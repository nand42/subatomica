%Zarantonello Umberto 2021-04-30

\section{Applicazioni}
\subsection{Risonanza Magnetica Nucleare}
La risonanza magnetica è una tecnica di imaging medico non invasiva.
Si basa sul fatto che ad ogni protone è associato un piccolo campo magnetico associato al dipolo. 
Ponendo questi dipoli sotto l'azione di un campo magnetico forte, si avrà l'allineamento con il campo, sia nella stessa direzione che i direzione opposta.
\begin{equation}
E=-\bar \mu\cdot \bar B
\end{equation}
Ciò che si otterrà è che l'energia sarà minore se il campo e il dipolo sono solidali ($\mu\uparrow\  B\uparrow$), mentre sarà maggiore nel caso contrario ($\mu\uparrow\  B\downarrow$).
\begin{figure}

\end{figure}
La differenza di energia è pari a $\Delta E=\mu B$.
La risonanza in campo magnetico sfrutta esclusivamente nucleoni ma in altri ambiti possono essere usati anche gli elettroni e dalla formula sopra si può vedere come sia diversa nei due casi.

I valori del magnetone di Bohr e del magnetone nucleare sono 
\begin{equation}
\begin{split}
\mu_B=\frac{e\hbar}{2m_e}=9,3\times10^{-24}\frac{j}{T}=5,8\times10^{-5}\frac{eV}{T}\\
\mu_N=\frac{e\hbar}{2m_p}=5,0\times10^{-27}\frac{j}{T}=3,8\times10^{-8}\frac{eV}{T}
\end{split}
\end{equation}

La risonanza magnetica funziona che una volta applicato un campo magnetico i livelli energetici con spin up e down si dividono e fornendo una radiofrequenza con energia pari alla differenza di energia generata il mio campione assorbirà energia, e io sono sensibile all'energia assorbita. 

Dato un elettrone se applico un campo B=0,335T, ottengo che la differenza di energia sarà pari a $\Delta E=6,22\times 10^{-24}j$.
In questo caso la frequenza coinvolta (assorbita dall'elettrone) è pari a $\nu =9,4GHz$, ovvero nel campo delle microonde.

Nel caso del protone, ovvero quello usato in campo medico, c'è la necessità di applicare un campo magnetico molto più alto per portare ad una separazione sufficientemente visibile. 
Applicando un campo $B=2T$ ottengo $\Delta E=5,64\times10^{-26}j$, vi è quindi la necessità di una radiofrequenza dell'ordine di $\nu =\frac{\Delta E}{h}=85MHz$.

Quello che succede è che pongo il paziente in questi campi magnetici e faccio passare una radiofrequenza. 
L'assorbimento di questa radiofrequenza avviene in funzione della densità del materiale biologico.
Una cosa di cui devo tener conto è l'agitazione termica dei protoni, posso immaginare che maggiore è l'agitazione termica minore sarà il contrasto dell'immagine, una soluzione sarebbe abbassare la temperatura ma non potendo congelare il paziente non resta che sfruttare campi magnetici molto forti($\Delta E/KT$).
Il segnale che io rivelo è proporzionale alla densità di protoni, le zone più chiare sono quelle a più elevata densità di protoni.
Per avere una risoluzione spaziale, ovvero una tridimensionalità, applico un gradiente di campo magnetico minimo che produrrà una risonanza diversa a diverse frequenze di radiofrequenza generando una variazione ulteriore che mi permette di ampliare l'imaging.
Per andare ad aumentare il contrasto andrò a vedere la risposta temporale del campione.
Quando irradio il tessuto i protoni assorbendo energia passeranno da uno stato all'altro (spin-flip), se poi io fermo la radiazione e lascio il tessuto tornare allo stato fondamentale potrò poi misurare il tempo impiegato per la diseccitazione e avrò quindi informazioni addizionali sul tessuto che circonda il materiale in analisi.

