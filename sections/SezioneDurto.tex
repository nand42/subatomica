% commento

\section{Sezione d'urto}


\subsection{Esperimento di Rutherford}
L'intuizione di Rutherford fu di utilizzare il decadimento dei nuclei $\alpha$ ed adottando un approccio statistico per ovviare al problema di non conoscere la posizione esatta delle particelle.

\paragraph{Elemento X} ha un numero di massa $A$ che corrisponde alla somma di \emph{neutroni} e \emph{protoni} nel nucleo ed un numero atomico $Z$ che è il numero di \emph{protoni} nel nucleo, per cui si scrive:
$$^{A}_{Z} X$$
in un atomo neutro il numero atomico corrisponde anche al numero di \emph{elettroni}.

Nell'esperimento, Rutherford, utilizza un nucleo di \textbf{Radio} (Ra) con numero di massa $A=226$ e numero atomico $Z=88$, per cui si scrive:
$$^{226}_{88} Ra$$
Il \emph{decadimento} che avviene si scrive nel modo seguente
\begin{equation}
^{226}_{88} Ra \longrightarrow ^{222}_{86}Rn + ^{4}_{2}He + Q
\end{equation}
Nella reazione si conserva il numero di massa totale $ 226 = 222 + 4 $ e si conserva la carica totale $ 88 = 86 + 2 $;
$Q$ è il calore emesso dalla reazione esotermica/spontanea, ovvero energia rilasciata nel decadimento trasferita alla particella $\alpha$, sotto forma di energia cinetica, equivalente alla differenza di massa iniziale e finale.



