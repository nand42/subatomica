%% commento iniziale
%% anche su più righe ma vicine


% ogni file = un capitolo 

\section{Convenzioni LaTex} %inizia il capitolo con "section"
Iniziare a scrivere subito sotto al comando section per dare un senso estetico al codice.
Andare a capo ad ogni "punto", ogni frase inizia una nuova riga di codice.
Lasciando uno spazio bianco (due volte "invio") si ottiene un nuovo inizio paragrafo.

Tipo questo, che non sempre è bello da vedere mentre altre volte è molto utile.
Il concetto è: scrivere il codice il più attaccato possibile ma con maggior leggibilità possibile.

Le equazioni si possono scrivere in vari modi, ma i migliori, per compatibilità tra versioni sono:
\begin{enumerate}
\item equazione nel testo tipo $\cos \alpha = \frac{\pi}{2 \pi}$

\item equazione a capo a centro pagina e su una riga, senza numero di equazione
$$ \frac{d\sigma}{d\Omega}=\frac{b(\theta)}{\sin\theta}\frac{db}{d\theta} $$

\item equazione a capo a centro pagina e su una riga, con numero di equazione 
\begin{equation}
\Delta p_x \Delta x \ge \frac{\hbar}{2}
\end{equation}

\item equazione a capo a centro pagina su più righe, con numero di equazione
\begin{equation}
\begin{split}
\mbox{\underline{Maxwell-Boltzmann}}  \quad\quad  \frac{n_s}{g_s} & = \frac{1}{e^{\alpha + \beta E_s}} \\
\mbox{\underline{Bose-Einstein}}  \quad\quad  \frac{n_s}{g_s} & = \frac{1}{e^{\alpha + \beta E_s} - 1} \\
\mbox{\underline{Fermi-Dirac}}  \quad\quad  \frac{n_s}{g_s} & = \frac{1}{e^{\alpha + \beta E_s} + 1 } 
\end{split}
\end{equation}

\end{enumerate}



\begin{equation}
\begin{split}
a & = b \\
c & = a dahebfkjdnsdv
\end{split}
\end{equation}



\begin{equation}
\begin{split}
V & = \frac{3\pi}{\alpha} \\
& = \frac{^{ ds }}{sdf}
\end{split}
\end{equation}













