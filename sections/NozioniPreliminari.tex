\begin{center}
\begin{huge}
Nozioni Preliminari
\end{huge}
\end{center}
\vspace{1cm}

\paragraph{Elemento X} ha un numero di massa $A$ che corrisponde alla somma di \emph{neutroni} e \emph{protoni} nel nucleo ed un numero atomico $Z$ che è il numero di \emph{protoni} nel nucleo, per cui si scrive:
$$^{A}_{Z} X$$
in un atomo neutro il numero atomico corrisponde anche al numero di \emph{elettroni}.

\paragraph{Quantità utili}
\begin{itemize}
\item il fermi $1fm=10^{-15}m$
\item megaelettronvolt $MeV=1.6\times 10^{-13}J$
\item spesso si utilizza la normalizzazione di h tagliato e della velocità della luce $\hbar=c=1$  questo comporta il fatto che a volte la massa venga definita direttamente in MeV
\item $\hbar c=197.3MeV\cdot fm=1$
\item massa protone $m_pc^2=938.27MeV$
\item massa neutrone $m_nc^2=939.56 MeV$
\item massa elettrone $m_ec^2=0.511MeV$
\item unità di massa atomica $uc^2=931.5MeV$
\item costante di struttura fine $\alpha=\frac{e^2}{4\pi\varepsilon_0\hbar c}=\frac{1}{137}$
\item coincidenza tra energia e temperatura \[E=K_B T \hspace{0.5cm} K_B=1,38 \times 10^{-23}J/K\] per ricordare semplicemente questa quantità si ha che \[0.025eV=1/40eV=300K\]
\end{itemize}



\newpage
